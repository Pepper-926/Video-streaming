\documentclass[12pt]{article}
\usepackage[utf8]{inputenc}
\usepackage[spanish]{babel}
\usepackage{geometry}
\usepackage{amsmath}
\usepackage{hyperref}
\usepackage{titlesec}
\usepackage{fancyhdr}
\usepackage{enumitem}

\geometry{a4paper, margin=2.5cm}
\titleformat{\section}[block]{\bfseries\Large}{}{0pt}{}
\titleformat{\subsection}[block]{\bfseries\large}{}{0pt}{}

\title{Flixy: Plataforma de Video Streaming Educativo}
\author{Gabriel García Cortés, Anette Ruiz Saucedo, Daniel Alejandro Alvarez Mata \\
\small Profesora: Diana Lourdes Ávila Molina \\
\small Ingeniería de Software – 4to semestre \\
\small 11 de mayo de 2025}
\date{}

\pagestyle{fancy}
\fancyhf{}
\rhead{Flixy}
\lhead{Ingeniería de Software}
\cfoot{\thepage}

\begin{document}

\maketitle
\tableofcontents
\newpage

\section{Resumen Ejecutivo}
El proyecto consiste en una plataforma de videos educativos enfocada en tecnología, programación y desarrollo de software. Permite a los usuarios subir, visualizar y comentar videos, así como calificarlos y filtrarlos según sus intereses. Cuenta con un sistema de moderación de contenido, reproductor integrado y autenticación segura. Su propósito es facilitar el aprendizaje tecnológico y fomentar la interacción entre una comunidad de estudiantes, desarrolladores y entusiastas del software.

\section{Definición del Problema}
La plataforma resuelve el problema del acceso limitado a contenido educativo especializado en tecnología y programación, al centralizar recursos audiovisuales de calidad y permitir que los usuarios aprendan de forma interactiva. También aborda la falta de espacios seguros para compartir conocimientos mediante un sistema de moderación y roles de usuario, promoviendo así una comunidad colaborativa y confiable para el aprendizaje digital.

\section{Justificación}
Es necesario resolver este problema porque la demanda de habilidades tecnológicas crece constantemente y muchas personas no tienen acceso a educación de calidad en este campo. Los métodos tradicionales suelen ser poco dinámicos o inaccesibles, y la falta de una plataforma centralizada dificulta encontrar contenido confiable y actualizado. Además, el aprendizaje efectivo requiere interacción y retroalimentación, algo que muchas plataformas no ofrecen.

\section{Usuarios Objetivo}
\begin{itemize}
  \item \textbf{Usuarios generales:} Personas interesadas en aprender sobre tecnología y programación mediante videos educativos.
  \item \textbf{Creadores de contenido:} Usuarios con permisos para subir videos educativos, agregar descripciones y configurar su visibilidad.
  \item \textbf{Administradores:} Encargados de revisar, aprobar o rechazar contenido (videos y comentarios).
\end{itemize}

\section{Objetivos}
\begin{enumerate}
  \item Facilitar el acceso a contenido educativo de calidad en formato audiovisual sobre temas tecnológicos.
  \item Fomentar la creación de comunidad y la interacción entre usuarios a través de comentarios y calificaciones.
  \item Garantizar la seguridad y fiabilidad de la información y los servicios ofrecidos.
  \item Proveer búsqueda eficiente y filtrado avanzado de contenido.
\end{enumerate}

\section{Requisitos Funcionales y No Funcionales}
\subsection*{Requisitos Funcionales}
\begin{itemize}[leftmargin=*]
  \item Gestión de usuarios (registro, inicio de sesión, recuperación de contraseña, roles).
  \item Subida de videos con título, descripción, etiquetas, visibilidad.
  \item Visualización con reproductor integrado y recomendados.
  \item Comentarios y calificaciones con moderación.
  \item Búsqueda avanzada por filtros y palabras clave.
  \item Moderación de contenido por parte de administradores.
\end{itemize}

\subsection*{Requisitos No Funcionales}
\begin{itemize}[leftmargin=*]
  \item Rendimiento ágil y concurrencia.
  \item Escalabilidad.
  \item Seguridad: HTTPS, protección contra ataques, gestión de sesiones.
  \item Usabilidad y diseño responsivo.
  \item Mantenibilidad del código.
  \item Alta disponibilidad y respaldos periódicos.
\end{itemize}

\section{Casos de Uso Principales}
\begin{itemize}
  \item Registrarse / Iniciar sesión
  \item Visualizar videos
  \item Comentar y calificar
  \item Subir video
  \item Búsqueda avanzada
  \item Moderación de contenido
\end{itemize}

\section{Tecnologías Utilizadas}
\begin{itemize}[leftmargin=*]
  \item \textbf{Backend:} Django 5.2
  \item \textbf{Frontend:} No especificado
  \item \textbf{Base de datos:} PostgreSQL
  \item \textbf{Almacenamiento de archivos:} Amazon S3
  \item \textbf{Servidor web:} Gunicorn
  \item \textbf{Entorno de desarrollo:} Docker (no en producción)
\end{itemize}

\section{Consideraciones de Seguridad}
\begin{itemize}[leftmargin=*]
  \item Autenticación mediante JWT
  \item CSRF tokens activos
  \item Protección contra XSS e inyección SQL con mecanismos de Django
  \item HTTPS obligatorio en producción
\end{itemize}

\section{Pruebas y Evaluación}
Se han realizado pruebas en el flujo completo de uso: registro, login, subida de videos, visualización (tras aprobación), y comentarios (solo si está logueado). Aún falta implementar la búsqueda avanzada y la moderación de comentarios.

El sistema será considerado listo cuando todos los casos de uso funcionen correctamente en su mejor y peor escenario.

\section{Distribución del Trabajo}
\begin{itemize}
  \item Gabriel García Cortés – Base de datos
  \item Anette Ruiz Saucedo – Backend
  \item Daniel Álvarez Mata – Frontend
\end{itemize}

\section{Reflexión Final}
Como equipo, hemos aprendido a trabajar colaborativamente y aplicar buenas prácticas de documentación, planificación, desarrollo y seguridad. Los principales retos fueron definir el alcance realista, elegir tecnologías adecuadas, coordinar el trabajo y equilibrar experiencia de usuario con seguridad.

Si comenzáramos de nuevo, dedicaríamos más tiempo a la planificación, usaríamos una mejor estructura de carpetas desde el inicio y profundizaríamos en herramientas como Docker y conceptos como WSGI/ASGI.

\end{document}
